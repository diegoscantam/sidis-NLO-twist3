\chapter{Hadronic Matrix Elements}\label{chap:matrixelements}
\pagenumbering{arabic}
\setcounter{page}{1} 
Describing high-energy scattering processes involving hadrons is an outstanding challenge in modern particle physics. Hadrons are composite particles made up of quarks and gluons (collectively called partons), which are the fundamental constituents of matter according to the theory of quantum chromodynamics (QCD). Understanding the behavior of hadrons in high-energy collisions therefore requires a deep understanding of the inner structure of these composite particles, how they form and how they interact. However, due to the non-perturbative nature of QCD at low energies, directly calculating hadronic properties from first principles is extremely challenging. To overcome this difficulty, physicists have developed factorization theorems that separate the short-distance, perturbative partonic interactions from the long-distance, non-perturbative hadronic structure. This separation allows us to express cross sections for high-energy processes as convolutions of perturbatively calculable hard scattering coefficients and non-perturbative matrix elements that encode the hadronic structure. This separation works because QCD is an asymptotically free gauge theory, meaning that the strong interaction becomes perturbative at sufficiently small distances, with an effective coupling that becomes weaker and weaker with higher energies.

In order to obtain a factorized spin-dependent cross section for semi-inclusive deep inelastic scattering, we will need several so called \textit{soft hadronic matrix elements}. The "nucleon-to-parton" process, as well as the "parton-to-hadron" hadronization phenomenon, are typically described through non-perturbative matrix elements of certain QCD operators. In principle, they contain all relevant information about the distribution of quarks and gluons inside the nucleon, as well as the decay process from "free" quarks and gluons into color-neutral states such as baryons and mesons. The inner hadronic structure is therefore encoded in these non-perturbative objects. Since they are non-perturbative, they cannot be computed from first principles in QCD. However, they can be extracted from experimental data CITE or computed using non-perturbative methods such as lattice QCD or QCD sum rules CITE. It is therefore clear that a thorough understanding of these matrix elements is essential in order to study high-energy scattering processes involving hadrons. For the study of unpolarized cross sections, the relevant matrix elements are closely related to the so called parton distribution functions (PDFs) and fragmentation functions (FFs). PDFs describe the distribution of partons inside a hadron, while FFs describe the hadronization of partons into hadrons. Both PDFs and FFs can be classified according to their twist, which is a measure of their relevance in high-energy processes. Leading-twist PDFs and FFs are the most important ones, as they dominate the cross section at high energies. However, sub-leading twist PDFs and FFs also play a significant role in certain processes, especially when spin degrees of freedom are involved. 

In this chapter, after a short introduction on the concept of twist, we will introduce the relevant twist-3 matrix elements that will be used in the following chapters to describe non-perturbative effects. Some important relations among these functions, crucial in our analysis, are also presented.

\section{What is twist anyway?}
In the context of hadronic observables, the term \textit{twist} often appears when classifying parton distributions and fragmentation functions. It finds its formal definition through a very powerful field theoretical tool: the operator product expansion (OPE). We will quickly discuss this formal viewpoint, and soon move on to a more practical and quantitative redefinition.

The OPE is a general method in quantum field theory (QFT) that allows us to express the product of two local operators at nearby spacetime points as a sum of local operators multiplied by coefficient functions. This expansion is particularly useful when dealing with non-local operators, which often arise in the context of high-energy scattering processes. Originally, a very successful application of the OPE concerns the treatment of the bilocal operator appearing in the hadronic tensor in deep inelastic scattering (DIS) CITE. It is
\begin{equation}\label{eq:WmunuDIS}
    W_{\mu\nu} = \int \dd^4 \xi e^{i q\xi} \mel{P}{\comm{J_\mu(\xi)}{J_\nu(0)}}{P}
\end{equation}
where $J_\mu(\xi)$ and $J_\nu(0)$ are the currents associated with the interaction where spin and color labels are omitted. It is important to point out that in the high-virtuality limit $Q^2\to \infty$, the hadronic tensor is dominated by $\xi^2\to 0$. This can be shown by regular stationary point methods in complex analysis CITE. Making use of this, one can expand around this point giving the OPE expression
\begin{equation}\label{eq:OPE}
    \comm{J_\mu(\xi)}{J_\nu(0)}\sim \sum_{\Theta} C_{\Theta}(\xi^2)\, \xi^{\mu_1}\cdots \xi^{\mu_{n_\Theta}}\,\Theta_{\mu_1 \dots \mu_{n_\Theta}}(0),
\end{equation}
where $\Theta_{\mu_1\dots\mu_n}(0)$ are now \textit{local} operators and the $C_\Theta(\xi^2)$ are complex-valued coefficients that can be ordered by their degree of singularity around $\xi^2= 0$. If we were to truncate the series, this would be accurate for small spacetime separations. Note that this nothing else but a QFT generalization of the Laurent series in complex analysis. In fact, taking the product of two holomorphic functions $f(z)$ and $g(w)$, one can always expand the product in the limit where the two complex numbers approach each other $z\to w$. This simply gives
\begin{equation}
    f(z)g(w)\sim \sum_{n=-\infty}^\infty c_n (z-w)^n,
\end{equation}
where, again, the coefficients $c_n$ are complex coefficients that can ordered by their degree of singularity. In this sense, the OPE is nothing particularly surprising but yet it finds countless applications in very different fields of QFT whenever non-local operators appear in our calculations CITE. This being said, we can plug Eq. \eqref{eq:OPE} into our original expression for the hadronic tensor in Eq. \eqref{eq:WmunuDIS} and obtain that the hadronic matrix element has the general form
\begin{equation}
 \mel{P}{\Theta_{\mu_1 \dots \mu_{n_\Theta}}(0)}{P}= P_{\mu_1} \cdots P_{\mu_{n_\Theta}} \,M^{d_\Theta - n_\Theta - 2}\, f_\Theta + \cdots .
\end{equation}
Note that this decomposition can be made by solely considering the Lorentz structure of the matrix element. Furthermore, the mass terms appear by dimensional analysis alone and must be there to match the operator dimensions. Typically, one identifies these $M$ factors with the typical hadronic mass scale in QCD, i.e. $M\sim\Lambda_{QCD}$. We finally call \textit{twist} of the operator $\Theta$ the power with which $M$ occurs
\begin{equation}
    t_{\Theta} \equiv d_\Theta - n_\Theta,
\end{equation}
where $d_\Theta$ is the dimension and $n_\Theta$ the spin of the operator. In general terms, an operator of dimension $d$ leads to a coefficient function in the OPE of the currents that scales as $\xi^{-6+d-s}$, obtained again via dimensional analysis. Performing a Fourier transform, one sees that the OPE in momentum space will carry a supression factor $\sim(M/Q)^{t_\Theta-2}$. Following this simple arguments coming from dimensional analysis, one sees that operators with lower twist will dominate the hadronic tensor at high $Q^2$. In this sense, twist is a measure of the relevance of a given operator in high-energy processes. For this very reason, it is common jargon to refer to higher twist effects and power corrections interchangeably. We also note that the smallest possible value for QCD operators is $t=2$, which is called \textit{leading twist} or \textit{leading power}. Operators with $t=3$ are called \textit{sub-leading twist} (or {power}) or \textit{twist-3}, and so on. In the following, we will focus on twist-2 and twist-3 operators, as they are the most relevant for our analysis. Another important point is that as soon as hadronic spin degrees of freedom are involved, QCD operators must have at least $t=3$. This highlights the fact that, for spin observables, sub-leading twist factorization is not merely a more accurate description at the next power in $M/Q$, but it is rather essential to describe any non vanishing effect.
\begin{figure}
    \centering
    \includegraphics[width=0.85\linewidth]{fig/lightcone.pdf}
    \caption{Pictorial representation of the decomposition on the two light cone directions ($n_+^\mu$ and $n_-^\mu$) and on the transverse space.}
    \label{fig:lightcone}
\end{figure}
ADD LINK TO PDF/FF!!!!



So far we have given a formal definition of twist based on the OPE. However, there is yet another way to define twist. This alternative definition is based on the decomposition of quark fields into so called "good" and "bad" components on the light cone. It turns out that the treatment of both perturbative and non-perturbative quantities in high-energy collisions is much simpler when using light-cone coordinates. This should not be particularly surprising, based on the fact that highly relativistic on-shell particles are described by light-like momentum four-vectors since masses are typically negligible. In fact, in processes where a large momentum scale is involved, such as the virtuality of the exchanged photon in deep inelastic scattering $\ell N\to\ell'X$, it is convenient to decompose four-vectors into components along two light-like directions. These directions are typically chosen to be (anti)aligned with the momenta of the incoming and outgoing hadrons in a given scattering process. In turn, this naturally implies the existence of a two-dimensional space which is transverse to both these two light-like directions. The usual Minkowski four-dimensional spacetime is then partitioned into a "plus" direction $n^{\mu}_+$, a "minus" direction $n_-^\mu$, and a transverse ($\perp$) space. The usual metric tensor $g^{\mu\nu}$ is then related to these directions as
\begin{equation}
    g^{\mu\nu}_\perp=g^{\mu\nu} - n_+^\mu n_-^\nu - n_-^\mu n_+^\mu,
\end{equation}
where $g_\perp^{\mu\nu}$ is the projector onto the transverse space. One can also decompose field operators into two components along the plus and minus directions using the projection operators for spinors
\begin{equation}
    \begin{aligned}
    \mathcal{P}_{\pm} &= \frac{1}{2}\gamma^{\mp}\gamma^{\pm}\\
    \gamma^{\pm} &=\frac{\gamma^0\pm\gamma^3}{\sqrt{2}},
    \end{aligned}
\end{equation}
satisfying the usual projector properties such as $\mathcal{P}_{\pm}^2=\mathcal{P}_\pm$ and $\mathcal{P}_{\pm}\mathcal{P}_{\mp}=0$. The light cone projections of a Dirac field will be then denoted as $\psi_+ = \mathcal{P}_+ \psi\equiv\phi$ and $\psi_- = \mathcal{P}_- \psi \equiv \chi$ and they are called the "good" and "bad" light cone components of the field $\psi$, respectively. This decomposition is particularly insightful if we apply it to the QCD equation of motion
\begin{equation}\label{eq:dirac on LC}
    \begin{aligned}
        i \gamma^- D_- \chi &= \gamma^\perp \cdot D^\perp \phi + m\phi\\
        i \gamma^+ D_+ \psi &= \gamma^\perp \cdot D^\perp \chi + m\chi,
    \end{aligned}
\end{equation}
where $D$ is the usual covariant derivative $D_{\pm} = \pdv{\xi^{\pm}} -igA^{\mp}$. It is interesting to note that in light cone quantization the "time" evolution parameter is taken to be $\xi^+$. Therefore, fields are canonically quantized at equal $\xi^+$, rather than equal $\xi^0$ as in usual equal-time quantization. If we work in light cone gauge $A^+=0$, the first equation in Eq.\eqref{eq:dirac on LC} only involves $\pdv{\xi^-}$, suggesting that the bad light cone component $\chi$ is not really an independent dynamical field. In fact, the Dirac equation constrains $\chi$ to be expressed in terms of the good component $\phi$ and the gluon field $A^\perp$. This is a very important observation, since it implies that any hadronic matrix element involving bad components can be rewritten in terms of good components and gluon fields. To prove our point, let us now consider two examples of light cone correlation functions. Starting with the unpolarized twist-2 parton density $f_1(x)$, we can decompose the bilocal operator $\bar{\psi}(0)\slashed{n}\psi(\lambda n)$ on the light cone and perform some simple Dirac algebra, obtaining
\begin{equation}
    \begin{aligned}
    f_1(x)&=\frac{1}{2}\int\frac{\dd \lambda}{2\pi} e^{i\lambda x} \mel{N(P,S)}{\bar{\psi}(0)\slashed{n}\psi(\lambda n)}{N(P,S)}\\
    &=\frac{1}{\sqrt{2} k^+}\int\frac{\dd \lambda}{2\pi} e^{i\lambda x} \mel{N(P,S)}{\phi^\dagger(0)\phi(\lambda n)}{N(P,S)}.
    \end{aligned}
\end{equation}
We stress the crucial point that only good light cone components $\phi$ occur in the matrix element. If we instead repeat the same exercise for the twist-3 function $e(x)$, we get
\begin{equation}
    \begin{aligned}
    e(x)&=\frac{1}{2M}\int\frac{\dd \lambda}{2\pi} e^{i\lambda x} \mel{N(P,S)}{\bar{\psi}(0)\psi(\lambda n)}{N(P,S)}\\
    &=\frac{1}{2M}\int\frac{\dd \lambda}{2\pi} e^{i\lambda x} \mel{N(P,S)}{\phi^\dagger(0)\gamma^0\chi(\lambda n) + \chi^\dagger(0)\gamma^0\phi(\lambda n) }{N(P,S)}\\
    &=-\frac{1}{4Mx}\int\frac{\dd \lambda}{2\pi} e^{i\lambda x} \mel{N(P,S)}{\phi^\dagger(0)\gamma^0 \slashed{n}\slashed{D}_\perp(\lambda n)\phi(\lambda n) }{N(P,S)} + \text{h.c.},
    \end{aligned}
\end{equation}
where it is clear that also bad components $\chi$ arise. In the last line we eliminated the $\chi$ field with the Dirac equation in favor of a quark-gluon description. With this simple considerations, we stumbled upon a very important and insightful result: every factor of $\chi$ in the light cone decomposition of a correlation function contributes as an extra unit of twist to the associated matrix element. Similarly, the presence of transverse gluon fields contributes as an extra unit as well. It is as if $\phi$ and $D_\perp$ had twist one and $\chi$ had twist two. Schematically, it is
\begin{equation}
    \begin{aligned}
        \phi^\dagger \phi &\Leftrightarrow \text{twist-2}\\
        \phi^\dagger \chi\,\,\, \text{or}\,\,\, \phi^\dagger D_\perp \phi &\Leftrightarrow \text{twist-3}\\
        \chi^\dagger \chi\,\,\, \text{or}\,\,\, \phi^\dagger D_\perp  D_\perp \phi &\Leftrightarrow \text{twist-4}\\
        &\,\vdots
    \end{aligned}
\end{equation}

Concluding this introductory section about twist and light cone physics, we can make some important remarks:
\begin{itemize}
    \item The twist on an operator is formally defined via the dimension and the spin of the operator itself, in particular via its operator product expansion if the operator is non-local or bi-local. Thanks to dimensional analysis, one soon realizes that twist is also connected to the power in which $M/Q$ factors appear in the expansion. This is important not only on the theory side, but also on the phenomenological and experimental point of view. In fact, a sub-leading twist observable will be power suppressed, especially in high-energy collider experiments.
    \item From a theoretical viewpoint, hadronic spin degrees of freedom cannot be taken into account at leading twist. There is simply no way of parametrizing a spin-dependent hadronic matrix element without introducing any mass factor to match the dimensions of the QCD operators. Twist-3 factorization is therefore essential when it comes to studying these spin-dependent phenomena.
    \item Lastly, by decomposing the amplitudes on the light cone, one associates higher-twist effects with distributions that probe also bad light cone components of the fields. Equivalently, since these bad components are not independent fields by themselves, one can transform them away in favor of quark-gluon composites. In this very sense, sub-leading twist effects probe \textit{multi-parton correlations} within hadrons and gives us information about \textit{quark-gluon-quark} hadronic matrix elements.
\end{itemize}


\section{Multi-parton correlations}
In this section we will discuss in detail the non-perturbative objects appearing in our factorization formulae. We already mentioned that, on one hand, the "nucleon-to-parton" process is described via soft matrix elements of certain QCD operators on the light cone, which are intimately related to parton distribution functions. On the other hand, the "parton-to-hadron" phenomenon is encoded in fragmentation (or decay) functions, also expressed as hadronic matrix elements on the light cone. We also discussed that such bi-local operators can be expanded as a sum of local operators, in which the mass factors appearing in the coefficients gives us information about the twist of the operator. In this work, as many other in the literature, we will organize PDFs into what people call \textit{distribution correlators}, collectively denoted $\Phi(x)$. The term \textit{distribution} will therefore be used solely to refer to non-perturbative functions on the nucleon side. Analogously, FFs are collected in what we call \textit{fragmentation correlators}, often denoted with $\Delta(z)$. As anticipated, in this work we focus on the sub-leading twist effects only coming from the fragmentation process. Since at twist-3 level many conventions concerning matrix elements and functions are used, we believe is crucial that we clearly state which non-perturbative input is employed in our calculation.

The correlators, except when the parton content is purely gluonic, are matrices in Dirac space or, to be specific, Dirac bilinears. A neat way to express the parametrization of these matrix elements is writing them as a sum over the basis Dirac matrices. One can achieve this by performing a so-called Fierz transform. For a distribution correlator $\Phi_{ij}$, it is
\begin{equation}
    \begin{aligned}
            \Phi_{ij}&=\sum_\Gamma \Phi^{[\Gamma]} \,\bar{\Gamma}_{ij}\\
            \text{with}\quad \Phi^{[\Gamma]}&=\frac{1}{2}\Tr\left[\Phi\,\Gamma\right]\\
            \Gamma&\in\left\{ \gamma^\mu,\gamma^\mu\gamma_5, i \sigma^{\mu\nu}\gamma_5,1,i\gamma_5\right\}\\
            \bar{\Gamma}&\in\left\{ \frac{1}{2}\gamma^\mu,-\frac{1}{2}\gamma^\mu\gamma_5, -\frac{i}{4} \sigma^{\mu\nu}\gamma_5,\frac{1}{2}1,-\frac{i}{2}\gamma_5\right\},
    \end{aligned}
\end{equation}
and analogously for the fragmentation correlator $\Delta_{ij}$. Interestingly, the traces $\Phi^{[\Gamma]}$ and $\Delta^{[\Gamma]}$ turns out to be proportional to the various PDFs and FFs, depending on the specific Dirac structure $\Gamma$. In this way, we see that the correlators can be written as a sum of Dirac matrices, where each structure comes with a different non-perturbative parametrization. This is what we mean by saying that PDFs and FFs are "organized" into these more general onjects.

With this notation in mind, we recap the relevant correlators in the next subsections.

\subsection{Distribution functions}
Starting with the nucleon side, the tratment of PDFs is not particularly tedious since we only deal with leading twist distributions.
\subsubsection*{Leading twist}
The relevant hadronic matrix element describing the "nucleon-to-quark" process is given by the so-called quark-quark distribution correlator
\begin{equation}
    \begin{aligned}
        \Phi^q_{ij}(x)&= \int\frac{\dd \lambda}{2\pi} \,e^{i\lambda x}\mel{N(P,S)}{\bar \psi^q_j(0) \psi^q_i(\lambda n)}{N(P,S)}\\
        &= \frac{1}{2}\left(\slashed{P}\right)_{ij} f^q_1(x)
        -\frac{1}{4} \left(\comm{\slashed{P}}{\slashed{S}}\gamma_5\right)_{ij} h^q_1(x)+\cdots .
    \end{aligned}
\end{equation}
The different Dirac structures appearing in the correlator are parametrized by the parton distribution functions $f_1(x)$ and $h_1(x)$, which are referred to as the leading-twist unpolarized and transversity parton densities respectively. The former enters the unpolarized observables, and it has been extensively determined by numerous experimental collaborations throughout the years CITE. In other words, $f_1(x)$ is arguably the most constrained and the best understood PDF among all non-perturbative functions in hadronic physics. The latter, on the other hand, is present only if the nucleon is transversely polarized. Although being a leading twist effect, transversity has been known with satisfactory accuracy only in recent years \cite{Gamberg2022Htilde}. This is mainly linked to the fact that it is experimentally challenging to set up and control transversely polarized hadronic targets. This is even more true for transversely polarized beams, which is currently one of the experimental frontiers when it comes to the field of hadronic spin physics.

One may also have an equivalent hadronic matrix element describing the "nucleon-to-gluon" process, also often appearing in hadronic observables. It is described through the gluon-gluon distribution correlator CHECK
\begin{equation}
    \begin{aligned}
        \Phi^{g,\mu\nu}(x)&= \int\frac{\dd \lambda}{2\pi} \,e^{i\lambda x}\mel{N(P,S)}{F^{n\nu}(0)F^{n\mu}(\lambda n)}{N(P,S)}\\
        &= \frac{1}{2}g_T^{\mu\nu} {\color{red} x } f^g_1(x)+\cdots,
    \end{aligned}
\end{equation}
where $f_1^g(x)$ is the leading-twist unpolarized gluon distribution function. Again, the gluonic contribution entering the unpolarized PDFs is also well understood. As we will see later, in semi-inclusive deep inelastic scattering this correlator only appears at next-to-leading order in the strong coupling. This is because the gluon coming from the nucleon cannot directly couple to the virtual photon, and hence an additional partonic subprocess of order $\mathcal{O}(\alpha_S)$ must occur. In any case, it is quite useful to visualize these objects in a Feynman cut diagram notation \cite{handbookqcdsterman95,Collins_2011}. In fact, if one writes down the total squared amplitude $|\mathcal{A}|^2$ for a given hadronic scattering process assuming the parton model, it is easy to realize that these soft hadronic matrix elements are indeed ubiquitous in high-energy physics. The two distribution correlators are shown in Fig.~\ref{fig:distribution corr}.

\begin{figure}[h]
    \centering
    \includegraphics[width=0.8\linewidth]{fig/phi.pdf}
    \caption{Quark-quark (left) and gluon-gluon (right) distribution correlators appearing at leading-twist.}
    \label{fig:distribution corr}
\end{figure}

\subsection{Fragmentation functions}
Moving on to the fragmentation process (which we describe at twist-3 level), things get slightly more involved. In fact, one typically classifies the different contributions into three main categories: \textit{intrinsic, kinematical} and \textit{dynamical} twist-3 functions. 

\subsubsection*{Intrinsic twist-3 fragmentation}
The first type of sub-leading twist hadronic matrix elements are the so called \textit{intrinsic} twist-3 functions. They are expressed in terms of 2-parton correlations, involving a pair of quark fields or a pair of gluons. The fact that only one parton (on each side of the cut) is participating in the fragmentation process is a common feature with leading twist correlations.\\
Starting with the quark-quark case, these quantities are organized into a quark-quark fragmentation correlator
\begin{equation}
    \begin{aligned}
        \Delta^q_{ij}(z)&= \sum_X\int\frac{\dd \lambda}{2\pi} \,e^{-i\frac{\lambda}{z}}\mel{0}{\psi^q_i(0)}{h(P_h);X}\mel{h(P_h);X}{\bar \psi^q_j(\lambda m)}{0}\\
        &= \frac{1}{z}\left(\slashed{P}_h\right)_{ij} D^q_1(z)-\frac{iM_h}{2z} \left(\comm{\slashed{P}_h}{\slashed{m}}\right)_{ij} H^q(z)+\frac{M_h}{z}(\mathbb{1})_{ij}\,E^q(z)+\cdots,
        \end{aligned}
\end{equation}
where again the dots denote terms that are irrelevant for the present analysis. The $D_1(z)$ term is the familiar twist-2 unpolarized fragmentation function, appearing in numerous different high-energy processes involving the production of unpolarized hadrons. The sub-leading twist fragmentation function $H(z)$ describes the (power suppressed) production of unpolarized hadrons in the final state. The fact that twist-3 effects are generally smaller compared to the leading-twist case can be easily spotted by the fact they come with a mass factor $M_h$, the hadron mass. Compared to any hard scale, which we consider to be at least of the order of a some $\rm GeV$'s, typical produced hadron masses range from a few hunder $\rm MeV$'s ($\pi$) all they way up to a few $\rm GeV$'s ($\Lambda$). In this sense, the effect is said to be power suppressed.\\
One can also build a matrix element in which the exchanged parton coming from the nucleon is not a quark but rather a gluon. In this case, we only have leading twist effects, described in terms of a gluon-gluon fragmentation correlator CHECK
\begin{equation}
    \begin{aligned}
        \Delta^{g,\mu\nu}(z)&= \sum_X\int\frac{\dd \lambda}{2\pi} \,e^{-i\frac{\lambda}{z}} CHECK\\
        &= \frac{1}{z} g_T^{\mu\nu}\left(\slashed{P}_h\right)_{ij} D^g_1(z)+\cdots,
        \end{aligned}
\end{equation}
where $D_1^g(z)$ is the unpolarized gluon fragmentation function.

Similarly to distribution matrix elements, also fragmentation correlators are conveniently depicted as cut diagrams, as shown in Fig.~\ref{fig:fragmentation corr I}.
\begin{figure}[h]
    \centering
    \includegraphics[width=0.8\linewidth]{fig/delta_i.pdf}
    \caption{Soft matrix elements relevant for leading twist and intrinsic twist-3 fragmentation. The quark-quark correlator (left) can depict both $D_1(z)$ or $H(z)$ depending on the Dirac structure assumed. The gluon-gluon correlator is also shown (right).}
    \label{fig:fragmentation corr I}
\end{figure}

\subsubsection*{Kinematical twist-3 fragmentation}
Another kind of two-parton correlation functions are the so called \textit{kinematical} twist-3 functions. Contrary to the intrinsic twist-3 case, these objects incorporate (indirectly) also the transverse momentum dependence of partons. Instead of assuming the hadron and the fragmenting parton to be exactly collinear, now the kinematical approximation to the partonic momentum reads
\begin{equation}
    p^\mu = \frac{1}{z}P_h^\mu + p_T^\mu,
\end{equation}
where $p_T$ denotes the transverse partonic momentum. In practice, one performs a Taylor expansion to first order in $p_T$ in the hard scattering subprocess, commonly called the \textit{collinear expansion}. After this expansion, one often integrates out the transverse momentum, leaving us with a collinear matrix element. Keeping at first the full $p_T$-dependence into account, one may define a transverse momentum dependent (TMD) fragmentation correlator
\begin{equation}\label{eq:DeltaTMD}
    \begin{aligned}
        \Delta^q_{ij}(z,p_T)&= \sum_X\int\frac{\dd \lambda}{2\pi}\int\frac{\dd^{d-2}\eta_T}{(2\pi)^2} \,e^{-i\frac{\lambda}{z}-i\eta_T\cdot p_T}\\
        &\qquad\times\mel{0}{\psi^q_i(0)}{h(P_h);X}\mel{h(P_h);X}{\bar \psi^q_j(\lambda m+\eta_T)}{0}\\
        &= \frac{i z M_h}{2}\comm{\slashed{P}_h}{\slashed{m}}H^q(z,p_T)+\frac{iz}{2M_h}\comm{\slashed{p_T}}{\slashed{P}_h}H_1^{\perp,q}(z,p_T)\\
        &\quad +zM_h(\mathbb{I})_{ij}\,E^q(z,p_T)+\cdots  .
        \end{aligned}
\end{equation}
COMMENT ON FUNCTIONS + KINEMATICAL CONTRIBUTIONS FRM LORENTZ TRANSFORM\\
Now, the collinear matrix element relevant within the collinear twist-3 formalism is nothing else but the $p_T$-weighted correlator. It is
 \begin{equation}
    \begin{aligned}
        \Delta_{\partial,ij}^\rho(z)&= \int\dd^{d-2}p_T\,p_T^\rho\,\Delta_{ij}(z,p_T)\\
        &=\frac{z^{2\epsilon}}{z}\frac{i M_h}{2}\left(\comm{\slashed{P}_h}{\gamma^\rho}-\comm{\slashed{P}_h}{\slashed{m}}P_h^\rho\right)_{ij}H_1^{\perp(1)}(z)+ \cdots,
        \end{aligned}
\end{equation}
where the first moment of the Collins function is defined as the first $\vec{p_T}$ moment of the TMD fragmentation function CHECK Z!
\begin{equation}
    H_1^{\perp(1)}(z)=z^2\int \dd^2 p_T \frac{\vec{p_T}^2}{2 M_h^2}H_1^{\perp}(z,z^2 p_T^2).
\end{equation}

Kinematical twist-3 matrix elements are pictorially represented by the same kind of diagrams as the intrinsic twist-3 case. However, the kinematical approximation is different and one should keep in mind that the partonic momentum should have a transverse component, as shown in Fig.~\ref{fig:fragmentation corr K}

\begin{figure}[h]
    \centering
    \includegraphics[width=0.8\linewidth]{fig/delta_k.pdf}
    \caption{Soft matrix element relevant for kinematical twist-3 fragmentation.}
    \label{fig:fragmentation corr K}
\end{figure}

\subsubsection*{Dynamical twist-3 fragmentation}
The last type of twist-3 correlators are described in terms of matrix elements of three partonic fields, typically two quark/anti-quark fields and one gluonic field strength tensor. In general, such structures are composed of an interference of two transition amplitudes: one that is a coherent fragmentation of two partons into a hadron, and another that is the ordinary one-parton fragmentation. These objects are usually called \textit{dynamical} twist-3 correlation functions. Since the partons contributing to the hadron production are two, it makes sense that the matrix element will depend on more than one momentum fraction. The partonic momenta are now approximated as
\begin{equation}
    p^\mu = \frac{1}{z}P_h^\mu,\qquad p'^\mu = \frac{1}{z'}P_h^\mu.
\end{equation}
It is customary to introduce a scaled variable for the $z'$ momentum fraction,
\begin{equation}
    \zeta \equiv \frac{z}{z'},
\end{equation}
since it will simplify the treatment later on.

One type of dynamical twist-3 matrix elements are the so called \textit{quark-gluon-quark correlations}. In this case a quark and a gluon contribute to the final state hadronization process. The correlator is written as
\begin{equation}
    \begin{aligned}
        \Delta^{qg,\rho}_{F,ij}(z,\zeta)&= \sum_X\int\frac{\dd \lambda}{2\pi}\int\frac{\dd \eta}{2\pi} \,e^{-i\frac{\lambda }{z}\zeta-i\frac{\eta}{z}\left(1-\zeta\right)}\mel{0}{ig_S \mu^\epsilon m_\sigma G^{\sigma \rho}(\eta m)\psi_i(0)}{h(P_h);X}\\
        &\qquad \qquad \qquad \times\mel{h(P_h);X}{\bar \psi_j(\lambda m)}{0}\\
        &= z^{2\epsilon}\frac{iM_h}{2z}\left(\comm{\slashed{P}_h}{\gamma^\rho}-\comm{\slashed{P}_h}{\slashed{m}}P_h^\rho\right)_{ij}\,i\left(\hat H^{qg}_{FU}(z,\zeta)\right)^*+\cdots,
        \end{aligned}
\end{equation}
where $\hat{H}_{FU}^{qg}$ is the quark-gluon-quark fragmentation function describing the production of unpolarized hadrons. It is important to stress that this matrix element contains a gluonic field strength tensor $G^{m\rho}$, rather than a simple gluon field operator. Actually, when writing down the amplitudes, one ends up with a version of the above matrix element, $\Delta^{qg}_{A,ij}(z,\zeta)$, with a gluon field alone instead of the field strength tensor. It turns out that the two matrix elements are related to each other. In fact, in light-cone gauge $m\cdot G=0$, the gluon field-strength tensor satisfies contracted with $m_\mu$, is nothing else but gluon field derivative along the good light-cone direction, i.e. $m_\mu G^{\mu\nu}\sim \pdv{\eta} G^\nu$. Hence, by simple integration by parts, one can show that two must be related to each other via the simple relation
\begin{equation}
    \Delta^{qg,\rho}_{A,ij}(z,\zeta)=\frac{(-i)}{1-\zeta}\Delta^{qg,\rho}_{F,ij}(z,\zeta)
\end{equation}
for sufficiently well-behaved gluon fields at the boundaries (or asymmetric boundary conditions). In passing, we mention that in twist-3 fragmentation only transverse gluons has to be taken into account in the end \cite{boer_universality_2003}, since 
\begin{equation}
    G^\mu=\underbrace{(m\cdot G)}_{=0}P_h^\mu+\underbrace{(P_h \cdot G)}_{\text{twist-4}}m^\mu+G_\perp^\mu\approx G_\perp^\mu=G_\nu g_\perp^{\mu\nu}.
\end{equation}
Our parametrization already takes into account this transverse Lorentz structure so we do not need to add this extra projection.

Another kind of dynamical twist-3 matrix element is given by a quark-antiquark-gluon correlation. Intuitively, the fragmentation is initiated from a quark-antiquark pair. The relevant correlator is CHECK
\begin{equation}
    \begin{aligned}
        \Delta^{q \bar{q},\rho}_{F,ij}(z,\zeta)&= \sum_X\int\frac{\dd \lambda}{2\pi}\int\frac{\dd \eta}{2\pi} \,e^{-i\frac{\lambda }{z}\zeta-i\frac{\eta}{z}\left(1-\zeta\right)}CHECK\\
        &\qquad \qquad \qquad CHECK\\
        &= z^{2\epsilon}\frac{iM_h}{2z}\left(\comm{\slashed{P}_h}{\gamma^\rho}-\comm{\slashed{P}_h}{\slashed{m}}P_h^\rho\right)_{ij}\,i\left(\hat H^{q\bar{q}}_{FU}(z,z')\right)^*+\cdots,
        \end{aligned}
\end{equation}
where $\hat{H}_{FU}^{q\bar{q}}$ is the quark-antiquark-gluon fragmentation function describing the production of unpolarized hadrons. Similarly to the previous case, we can define an analogous correlator with a gluon field instead of the gluon field-strength tensor. They are related via similar relations to the one presented above.

Very importantly, these kinds of fragmentation correlators have been shown to satisfy important constraints concerning their behavior on the support's boundaries. The first being that pole matrix elements vanish \cite{Meissner_2009}. More explicitly, it is
\begin{equation}
    \Delta_F(z,0)=\Delta_F(z,1)=0.
\end{equation}
Secondly, also the first derivatives must vanish when evaluated at the boundaries \cite{kanazawa_operator_2016}
\begin{equation}
    \left.\pdv{\Delta_F(z,\zeta)}{\zeta}\right\rvert_{\zeta\to0}=\left.\pdv{\Delta_F(z,\zeta)}{\zeta}\right\rvert_{\zeta\to1}=0,
\end{equation}
but not higher-order derivatives. Lastly, the diagram notation for these dynamical twist-3 correlations is presented in Fig.~\ref{fig:fragmentation corr D}
\begin{figure}[h]
    \centering
    \includegraphics[width=0.8\linewidth]{fig/delta_d.pdf}
    \caption{Soft matrix element relevant for dynamical twist-3 fragmentation. Quark-gluon-quark (right) and quark-antiquark-gluon (left) correlations are shown.}
    \label{fig:fragmentation corr D}
\end{figure}



\section{QCD equation of motion relations}
\noindent ADD LITTLE DERIVATION OF EOM IN D DIMENSIONS! RELEVANT FOR THIS WORK!\\

The previously introduced twist-3 fragmentation functions are not independent from each other. In fact, they are related through the QCD equation of motion. The corresponding constraint equations will be referred as \textit{equation of motion relations} (EoMRs). These relations between sub-leading twist matrix elements can be derived by applying the QCD EOM to the relevant quark fields appearing in the correlators. The quark fields satisfy the equation
\begin{equation}
    \left(i \slashed{D}(x) -m_q \right)\psi^q(x)=0
\end{equation}
where $D_\mu^{ab}(x)=\delta^{ab}\partial_\mu -ig_S G_\mu^{ab}(x)$ is the usual covariant derivative in the fundamental representation. Decomposing the above equation on the light-cone and considering the matrix elements appearing in the various correlators \cite{bacchetta_semi-inclusive_2007}, one can derive the following relations \cite{kanazawa_operator_2016}
\begin{equation}\label{eq:eomr H}
    \begin{aligned}
            (1-\epsilon)2z\mathcal{P}\int_0^1\dd \zeta\frac{\Im[\hat H^{qg}_{FU}(z,z/\zeta)]}{1-\zeta}&=H^q(z)+2z\,H_1^{\perp(1),q}(z)\\
            (1-\epsilon) (-2z)\,\mathcal{P}\int_0^1\dd \zeta\frac{\Re[\hat H^{qg}_{FU}(z,z/\zeta)]}{1-\zeta}&=E^q(z)-z\frac{m_q}{M_h}D_1^q(z)
    \end{aligned}
\end{equation}
Similar relations hold also for all the other twist-3 fragmentation functions, as well as for twist-3 parton distribution functions. Since other contributions do not enter our spin-dependent SIDIS cross section in the polarization case of interest, such EoMRs are irrelevant for the present work and we omit them here for brevity. At first glance, the above relation may seem redundant since it simply expresses the intrinsic and kinematical twist-3 fragmentation functions in terms of the dynamical one. However, as we will see in the next chapters, this relation is crucial to obtain a gauge invariant and infrared (IR) safe result for the final twist-3 cross section. In fact, when computing the hard scattering coefficients, one finds that they are not separately gauge invariant. Only after combining them according to the above EoMR, one obtains a gauge invariant result. This is a strong indication that the application of the QCD EoMR is not optional but it is rather an important and essential step in this kind of calculations.

\noindent We also note in passing that one can find yet another fragmentation function in the literature, often denoted as $\tilde H(z)$. This functions is nothing else but a shorthand for a particular combination of $H$ and $H_1^{\perp(1)}$ or, equivalently, an (integrated) $\hat H_{FU}$ quark-gluon-quark correlation function. It is defined as
\begin{equation}\label{eq:Htildedefinition}
    \tilde{H}^a(z)=2z\,\mathcal{P}\int_0^1\dd \zeta\frac{\Im[\hat H^a_{FU}(z,z/\zeta)]}{1-\zeta}=H^a(z)+2z\,H_1^{\perp(1)a}(z).
\end{equation}
It is worth mentioning that this specific fragmentation function will play an important role in Chap.~\ref{chap:numerics}, when some numerical explorations of the $A_{UT}^{\sin\phi_S}$ asymmetry will be performed. In fact, for the first time, this collinear quark-gluon-quark function $\tilde{H}(z)$ has been extracted only recently from experimental data \cite{Gamberg2022Htilde}.



\section{Lorentz invariance relations}
Interestingly, there is yet another set of relations among twist-3 fragmentation functions, which are derived from the requirement of Lorentz invariance. These relations, called \textit{Lorentz invariance relations} (LIRs), can be obtained by considering the most general decomposition of the relevant correlators in terms of all possible Lorentz structures. By imposing that the correlators transform correctly under Lorentz transformations, one can derive constraints on the fragmentation functions and parton distribution functions. In particular, one finds that certain combinations of twist-3 fragmentation functions must vanish. For our purposes, the most relevant LIR is \cite{kanazawa_operator_2016}
\begin{equation}
    \frac{H^a(z)}{z}=-\left(1-z\frac{\dd}{\dd z}\right)H_1^{\perp(1)a}(z)-2\int_0^1 \dd \zeta\frac{\Im[\hat H^a_{FU}(z,z/\zeta)]}{\left(1-\zeta\right)^2}\qquad \text{(LIR)}\,.
\end{equation}
Very similarly to the EoMR case, this relation can be essential to obtain a Lorentz invariant twist-3 cross section. In fact, a priori, twist-3 cross sections may be frame dependent, since the partonic coefficients depend on the light-cone vectors $n^\mu$ and $m^\mu$, which in turn depend on the choice of reference frame. As clearly seen in the previous paragraphs, the parametrization of twist-3 matrix elements depends explicitly on these vectors. The fact that twist-3 observables explicitly involve these light-cone vectors is intuitively clear. This is because any twist-3 observable is sensitive to the transverse component of some vector (e.g. the transverse spin component, etc...) and in order to define what "transverse" means one needs to introduce two light-cone vectors. As one sees, it is only after applying LIRs that one obtains a frame independent result and sees the manifest Lorentz invariance of the cross section. 
