
\chapter{Hard scattering coefficients}
\section{Unpolarized}
\label{appendix:unpolarized}
The hard scattering coefficients appearing in the unpolarized structure functions are given by
\begin{equation}
    \begin{aligned}
          &\mathcal{C}^{q\to q}_1 (w,v,\mu^2)= \left[C_F\left(-8-3\ln\frac{\mu^2}{Q^2}\right)\right]\delta(1-w)\delta(1-v)\\
        &\qquad+C_F\left[(1+v^2)\left(\frac{\ln(1-v)}{1-v}\right)_+ +1-v+(1+v^2)\frac{\ln v - \ln\frac{\mu^2}{Q^2}}{(1-v)_+}\right]\delta(1-w)\\
        &\qquad+C_F\left[(1+w^2)\left(\frac{\ln(1-w)}{1-w}\right)_+ +1-w+(1+w^2)\frac{-\ln w - \ln\frac{\mu^2}{Q^2}}{(1-w)_+}\right]\delta(1-v)\\
        &\qquad+C_F\left[\frac{2v^2w^2-2v^2w-2vw^2+4vw+v^2+w^2-2v-2w+2}{(1-w)_+(1-v)_+}\right]\\
        &\mathcal{C}^{g\to q}_1 (w,v,\mu^2)=T_F\left[(w^2+(1-w)^2)\left(\ln \frac{1-w}{w}-\ln\frac{\mu^2}{Q^2}\right)+2w(1-w)\right]\delta(1-v)\\
        &\qquad+T_F\left[\frac{(w^2+(1-w)^2)(v^2+(1-v)^2)}{v(1-v)_+}\right]\\
         &\mathcal{C}^{q\to g}_1 (w,v,\mu^2)= C_F\left[\frac{1+(1-v)^2}{v}\left(\ln \left(v(1-v)\right)-\ln\frac{\mu^2}{Q^2}\right)+v\right]\delta(1-w)\\
        &\qquad + C_F\left[\frac{1+v^2+w^2-2vw^2-2v^2w+2v^2w^2}{v(1-w)_+}\right]
    \end{aligned}
\end{equation}
and
\begin{equation}
    \begin{aligned}
          \mathcal{C}^{q\to q}_L (w,v,\mu^2)&= C_F\left[4 vw\right]\\
        \mathcal{C}^{g\to q}_L (w,v,\mu^2)&=T_F\left[8w(1-w)\right]\\
         \mathcal{C}^{q\to g}_L (w,v,\mu^2)&=C_F\left[4w(1-v)\right]
    \end{aligned}
\end{equation}

\section{Transversely polarized nucleon}\label{appendix:polarized}
The hard scattering coefficients appearing in the transverse spin structure functions are given by
\begin{equation}
    \begin{aligned}
          &\mathcal{C}^{q\to q}_1 (w,v,\mu^2)= \left[C_F\left(-8-3\ln\frac{\mu^2}{Q^2}\right)\right]\delta(1-w)\delta(1-v)\\
        &\qquad+C_F\left[(1+v^2)\left(\frac{\ln(1-v)}{1-v}\right)_+ +1-v+(1+v^2)\frac{\ln v - \ln\frac{\mu^2}{Q^2}}{(1-v)_+}\right]\delta(1-w)\\
        &\qquad+C_F\left[(1+w^2)\left(\frac{\ln(1-w)}{1-w}\right)_+ +1-w+(1+w^2)\frac{-\ln w - \ln\frac{\mu^2}{Q^2}}{(1-w)_+}\right]\delta(1-v)\\
        &\qquad+C_F\left[\frac{2v^2w^2-2v^2w-2vw^2+4vw+v^2+w^2-2v-2w+2}{(1-w)_+(1-v)_+}\right]\\
        &\mathcal{C}^{g\to q}_1 (w,v,\mu^2)=T_F\left[(w^2+(1-w)^2)\left(\ln \frac{1-w}{w}-\ln\frac{\mu^2}{Q^2}\right)+2w(1-w)\right]\delta(1-v)\\
        &\qquad+T_F\left[\frac{(w^2+(1-w)^2)(v^2+(1-v)^2)}{v(1-v)_+}\right]\\
         &\mathcal{C}^{q\to g}_1 (w,v,\mu^2)= C_F\left[\frac{1+(1-v)^2}{v}\left(\ln \left(v(1-v)\right)-\ln\frac{\mu^2}{Q^2}\right)+v\right]\delta(1-w)\\
        &\qquad + C_F\left[\frac{1+v^2+w^2-2vw^2-2v^2w+2v^2w^2}{v(1-w)_+}\right]
    \end{aligned}
\end{equation}
and
\begin{equation}
    \begin{aligned}
          \mathcal{C}^{q\to q}_L (w,v,\mu^2)&= C_F\left[4 vw\right]\\
        \mathcal{C}^{g\to q}_L (w,v,\mu^2)&=T_F\left[8w(1-w)\right]\\
         \mathcal{C}^{q\to g}_L (w,v,\mu^2)&=C_F\left[4w(1-v)\right]
    \end{aligned}
\end{equation}

\chapter{Splitting kernels}
\section{Unpolarized}
\section{Transversely polarized nucleon}


\clearpage

