\chapter{Semi-inclusive deep inelastic scattering at next-to-leading-order}

We perform our pQCD calculation at next-to-leading order in dimensional regularization in $d=4-2\epsilon$ dimensions. When it comes to renormalization, we adopt the common choice of using the minimal subtraction scheme ($\overline{\text{MS}}$).

\section{Leading-twist case}

\subsection{Virtual corrections}

The virtual graphs contributing at twist-2 level are shown in Fig.\ref{fig:Virt NLO tw2}
\begin{figure}
    \centering
    \includegraphics[width=0.75\linewidth]{fig/VirtNLOTw2.jpg}
    \caption{Virt NLO twist-2}
    \label{fig:Virt NLO tw2}
\end{figure}
\noindent Since we are performing our pQCD calculation in light-cone gauge, we shall employ an appropriate technique to evaluate loop integrals that allows us to treat the denominator of the gluon propagator in this gauge. Notoriously, this term requires a careful treatment since it leads to additional well-known \textit{light-cone divergences}. Not surprisingly, it turns out that a light-cone decomposition of the loop momentum allows for a systematic procedure to integrate out the $+$, $-$ and perpendicular components. Since this technique is essential for calculating virtual corrections also at the twist-3 level, we present it here in detail for the simpler leading-twist case. The method is quite general and can be readily extended to higher twists. Let's start by calculating the vertex correction graph. The amplitude reads
\begin{equation}\label{eq:amplitudevirtualtwist2}
\begin{aligned}
     &(\hat{\mathcal{M}}^{q \to q}_{\text{NLO,vir}})^\mu_{ij,ac} = \int\frac{\dd ^dr}{(2\pi)^d}\frac{\tilde N^\mu_{ij,ac}(r)}{[(p-r)^2+i\delta][(k-r)^2+i\delta][r^2+i\delta]r \cdot m},\\
       & \tilde N^\mu_{ij,ac}(r) \equiv -ig_S^2\mu^{4-d}T^\alpha_{ab} T^\beta_{bc}\delta_{\alpha\beta}\left[\gamma^\lambda(\slashed{p}-\slashed{r})\gamma^\mu(\slashed{k}-\slashed{r})\gamma^\eta\right]_{ij}\left[(r\cdot m)g_{\lambda\eta}-\kappa r_{(\lambda} m_{\eta)}\right].
    \end{aligned}
\end{equation}
As already anticipated, it is convenient to express the loop momentum in its light-cone decomposition $r^\mu = \alpha P_h^\mu+\beta m^\mu+r_\perp^\mu$ with $\alpha\equiv r\cdot m$ and $\beta\equiv r\cdot P_h$. By doing this, given a generic momentum $p_j$, the denominator of the propagators become of the form
\begin{equation}\label{eq:beta pole beta_j}
\begin{aligned}
        (p_j-r)^2+i\delta&=(2\alpha-2p_j \cdot m)(\beta - \beta_j),\qquad \beta_j\equiv -\frac{r_\perp^2+p_j^2-2\alpha \,p_j\cdot P_h+i\delta}{2\alpha-2p_j\cdot m}
\end{aligned}
\end{equation}
In order to simplify the calculation, it is convenient to perform the integration at the level of the hadronic tensor, and not just the amplitude alone. This is because many terms that are present in the hard scattering amlitude $\hat{\mathcal{M}}$ may vanish when traced with correlators and the LO interfering amplitude. We have then
\begin{equation}
\begin{aligned}
       (W^{q \to q}_{\text{NLO,vir}})^{\mu\nu} &=\int_{-\infty}^{+\infty}\frac{\dd \alpha}{2\pi} \frac{1}{2\alpha(2\alpha-2p \cdot m)(2\alpha-2 k \cdot m)\alpha}\int \frac{\dd^{d-2}r_\perp}{(2\pi)^{d-2}}\\
       &\times\int_{-\infty}^{+\infty}\frac{\dd \beta}{2\pi}\frac{N^{\mu\nu}(\alpha,r_\perp,\beta)}{(\beta-\beta_0)(\beta-\beta_1)(\beta-\beta_2)},\\
        N^{\mu\nu}&\equiv \frac{e_a^2 2x_B z_h }{Q^2}z_h^{-2\epsilon}\Tr[\tilde N^\mu\Phi^a\gamma^\nu \Delta^a].
\end{aligned}
\end{equation}
By studying the position of the poles $\beta_j$ ($j=0,1,2$) in the complex plane, we can use contour integration and the residue theorem to evaluate the $\beta$ integral in a straightforward manner. In order to perform the integration in this way, the integrand should fall off at least as $\sim\frac{1}{\beta}$ for large $\beta$ (meaning that $N^{\mu\nu}$ should be at most quadratic in $\beta$, which turns out to be the case). Evidently, the poles $\beta_j$ depend on $\alpha$, as seen in Eq.~\eqref{eq:beta pole beta_j}. In particular, the pole will lay above or below the real axis depending on the sign of $2\alpha-2p_j\cdot m$. By studying the imaginary part of all poles we can conveniently close the contour above or below the real axis and obtain the non-vanishing contributions to the $\beta$ integral. This, in turn, restricts the domain of integration over $\alpha$, since for certain values of $\alpha$ all the poles lie above (below) the real axis and we can close the contour below (above) and obtain zero since the curve does not include any poles. Going back to the integral above, we find that it is non-vanishing only for $0<\alpha<1/z_h$, and the only relevant pole is $\beta_1 = -(r_\perp^2+i\delta)/2(\alpha-1/z_h)$. A graphical sketch of the contour integration procedure can be found in Fig.~\ref{fig:beta contour}.
\begin{figure}
    \centering
    \includegraphics[width=0.99\linewidth]{fig/beta contour.pdf}
    \caption{Sketch of the light-cone contour integration technique used for calculating virtual graphs in this work. Poles placements are studied and non-vanishing contributions to the hadronic tensor are identified with the residue theorem.}
    \label{fig:beta contour}
\end{figure}
We then have
\begin{equation}
\begin{aligned}
       (W^{q \to q}_{\text{NLO,vir}})^{\mu\nu} &=\int_0^{1/z_h}\frac{\dd \alpha}{2\pi} \frac{1}{2\alpha(2\alpha-2p \cdot m)(2\alpha-2 k \cdot m)\alpha}\int \frac{\dd^{d-2}r_\perp}{(2\pi)^{d-2}}\\
       &\times 2\pi i\,\mathrm{Res}\left[\frac{N^{\mu\nu}(\alpha,r_\perp,\beta)}{2\pi(\beta-\beta_0)(\beta-\beta_1)(\beta-\beta_2)},\beta=\beta_1\right]
\end{aligned}
\end{equation}
After $\beta$ contour integration, the denominator factors arrange in the following way
\begin{equation}\label{eq:betaimbetaj=C(rt2+Del)}
    \beta_1-\beta_j=C_{1j}(\vec r_\perp^2+\Delta_{1j})
\end{equation}
with $j=0,2$ and $C_{1j},\Delta_{1j}$ some functions of $\alpha$. Then, the integration over the perpendicular loop momentum can be readily performed since the integrand depends only on the square modulus of $r_\perp$ (after dropping linear terms in the numerator since they vanish for symmetry). Adopting $d-2$ dimensional spherical coordinates, the angular integration is trivial and the radial integral turns out to be of the form
\begin{equation}
    \int_0^\infty \dd \rho \frac{ \rho^{2n+1-2\epsilon}}{(\rho^2+\Delta_1)(\rho^2+\Delta_2)}=\frac{\pi}{2\sin(\pi n-\pi\epsilon)}\frac{\Delta_1^{n-\epsilon}-\Delta_2^{n-\epsilon}}{\Delta_1-\Delta_2}\quad \text{if} \quad -1<n-\epsilon<1
\end{equation}
where $n\in \{\mathbb{N},0\}$ and assuming $\Delta_1,\Delta_2>0$. Hence, within this framework, divergences are regulated trough the dimension of the $d-2$ dimensional perpendicular space. Lastly, the $\alpha$ integral can be evaluated by direct integration. Typically, one ends up with integrand terms of the general form $\alpha^a (\alpha-1)^b (\zeta \alpha-1)^c$ which yields combinations of hyper-geometric and gamma functions. Interestingly, the gauge dependence trough the parameter $\kappa$ completely drops out in the end, leaving us with a gauge invariant expression. We therefore have the result
\begin{equation}\label{eq:NLO unpolarized virtual}
\begin{aligned}
      \frac{\dd \sigma^{UUU}_{\text{NLO,vir}}}{\dd x_B \dd y \dd \phi_l \dd z_h}&= \frac{\dd \sigma^{UUU}_{\text{LO}}}{\dd x_B \dd y \dd \phi_l \dd z_h}\times\frac{\alpha_S}{2\pi}  C_F S_\epsilon \left(\frac{\mu^2}{Q^2}\right)^{\epsilon}\left(-\frac{2}{\epsilon^2}-\frac{3}{\epsilon}-8\right)
\end{aligned}
\end{equation}
with the common $\overline{\text{MS}}$ scheme constant $S_\epsilon=(4\pi)^\epsilon/\Gamma(1-\epsilon)$. This in agreement with the orginal result for this vertex correction, first given in \cite{altarelli_large_1979}. \\
Now, we should also calculate two other diagrams, namely the self-energy corrections of the quark lines. Interestingly, if we repeat the very same procedure explained above, we find that after contour integration over $\beta$ the denominator factor is proportional to $r_\perp^2$ only ($\Delta=0$ in Eq.~\ref{eq:betaimbetaj=C(rt2+Del)}) for both diagrams. This leads to a contribution that goes as
\begin{equation}
     (\hat{\mathcal{M}}^{q \to q}_{\text{NLO,vir,self-energy}})^\rho\sim\int\frac{\dd^{d-2} r_\perp}{(2\pi)^{d-2}}\frac{A+B\,r_\perp^2}{r_\perp^2}
\end{equation}
which vanishes. This is because, in dimensional regularization, it is consistent to set to zero all scale-less integrals \cite{Schwartz:2014sze}. Therefore the vertex correction is the only virtual diagram that has to be taken into account. Interestingly, this diagram leads to an hadronic tensor which is IR divergent only. Therefore no direct UV counterterms are needed for this diagram (and for the two self-energy graphs, since they vanish). All in all, the full UV-renormalized virtual correction to the unpolarized cross section is just given by Eq.~\ref{eq:NLO unpolarized virtual}.

\subsection{Real corrections}
\textcolor{blue}{INSERT DIAGRAMS HERE}\\
Another $\mathcal{O}(\alpha_S)$ correction to the cross section involves the emission of a final unobserved parton produced in the hard scattering process. At NLO, not only we can have the $q \to q$ channel (with real emission of gluons), but also $q \to g$ and $g \to q$ (with real emission of quarks). For all channels, the hadronic tensor is modified since there is an additional integration over the phase-space of the unobserved parton. The integrations in the hadronic tensors are modified according to
\begin{equation}
\begin{aligned}W^{\mu\nu}_{\rm NLO, real}=\frac{e_a^2}{(2\pi)^{d-1}} \int \dd^dk\int \dd^dp\,\delta^+((q+k-p)^2)\Tr[\hat{\mathcal{M}}^\mu_{\rm NLO}\Phi^a \hat{\bar{\mathcal{M}}}^\nu_{\rm NLO}\Delta^a]
\end{aligned}
\end{equation}
where the momentum $r^\mu$ of the unobserved parton is set to $r=q+k-p$. We work again in the Breit frame as before. Integration over $\dd k^-$ and $\dd p^+$ is trivial. Here, since we are calculating the NLO corrections in the leading-twist unpolarized case, we can integrate over the transverse partonic momenta straight away. We are therefore left with integrations over the momentum fractions $x$ and $z$. With this conventions, we get \cite{koike_transverse_2022} 
\begin{equation}
\begin{aligned}
     \delta^+((q+k-p)^2)&=z z_h\delta\left(Q^2z_h^2\Big(1-\frac{z}{z_h} \Big)\Big(1-\frac{x}{x_B}\Big) - \vec P_{h\perp}^2\right)\theta(q^0+k^0-p^0)
\end{aligned}
\end{equation}
and the $\theta$ function restricts the kinematic variables in the following way
\begin{equation}\label{eq:NLOrealtw2_theta function condition}
    \begin{aligned}
        q^0+k^0-p^0\ge 0 & \iff \frac{x}{x_B}-\frac{z}{z_h}\left(1+\frac{Q_T^2}{Q^2}\right)\ge 0\iff \frac{x}{x_B}+\frac{z}{z_h}-2\ge 0
    \end{aligned}
\end{equation}
Note that this condition would result in a complicated convolution of the $x$ and $z$ integrals due to the fact that the integration limits may depend on the other variable, i.e. the lower integration limits would result in either $z_{\rm min}(x)$ or $x_{\rm min}(z)$. However, one can greatly simplify the problem by simply ignoring the target fragmentation region, introducing a kinematical cut $z_{\rm cut}$ and $x_{\rm cut}$ \cite{Sissakian_2004} In this work, we simply set the kinematical cut $z_{\rm cut}\approx z_h$ and $x_{\rm cut}\approx x_B$. With this prescription, the integration variables must satisfy $z>z_h$ and $x>x_B$ and our formulae match many results already available in the literature \cite{de_Florian_1998,Koike_2006}. We note however that one can also fully keep the integration limits $x_{\rm min}$ and $z_{\rm min}$ as they are, like done in \cite{kanazawa_contribution_2013}. Nothing really changes in the end, except for the fact that the integration limits are more complicated and numerically they would probably result in a slightly different overall cross-section value. The hadronic tensor now reads
\begin{equation}
\begin{aligned}
      W_{\mu\nu}^{\text{NLO,real}}&=\frac{e_a^2z_h}{(2\pi)^{d-1}} \int_{x_B}^{1} \dd x\int_{z_h}^{1}  \frac{\dd z}{z}\,\delta\left(Q^2z_h^2\Big(1-\frac{z}{z_h} \Big)\Big(1-\frac{x}{x_B}\Big) - \vec P_{h\perp}^2\right)w_{\mu\nu}
\end{aligned}
\end{equation}
where, depending on the channel, $w_{\mu\nu}$ assumes different forms
\begin{equation}
    \begin{aligned}
        w_{\mu\nu}^{a,q \to q}&=\Tr[\hat{\mathcal{M}}_\mu^{\text{NLO,real},q \to q}\Phi^a(x)\hat{\bar{\mathcal{M}}}_\nu^{\text{NLO,real},q \to q}\Delta^a(z)]\\
        w_{\mu\nu}^{a,g \to q}&=\Phi_g^{\lambda\eta}(x)\Tr[\hat{\mathcal{M}}_{\mu\lambda}^{\text{NLO,real},g \to q}\hat{\bar{\mathcal{M}}}_{\nu\eta}^{\text{NLO,real},g \to q}\Delta^a(z)]\\
        w_{\mu\nu}^{a,q \to g}&=\Delta_g^{\lambda\eta}(z)\Tr[\hat{\mathcal{M}}_{\mu\lambda}^{\text{NLO,real},q \to g}\Phi^a(x)\hat{\bar{\mathcal{M}}}_{\nu\eta}^{\text{NLO,real},q \to g}]
    \end{aligned}
\end{equation}
where $\Phi^{\mu\nu}_g(x)$ and $\Delta_g^{\mu\nu}(z)$ are the gluon distribution and fragmentation correlators respectively. Since $P_{h\perp}$ enters the expression only via its squared norm, the integration over the transverse momentum of the detected hadron can be written in spherical coordinates as
\begin{equation}
\begin{aligned}
    \int \dd^{d-2}P_{h\perp}&=\int_0^\infty \dd P_{h\perp}\, P_{h\perp}^{d-3}\int \dd \Omega_{d-2}=\int_0^\infty \dd  P_{h\perp}\,P_{h\perp}^{d-3} \frac{2 \pi^{\frac{d-2}{2}}}{\Gamma(\frac{d-2}{2})}\\&=\int \dd \vec P_{h\perp}^2\,P_{h\perp}^{d-4}\frac{ \pi^{\frac{d-2}{2}}}{\Gamma(\frac{d-2}{2})}=\frac{\pi^{1-\epsilon}}{\Gamma(1-\epsilon)}\int_0^\infty\dd \vec P_{h\perp}^2 \, (P_{h\perp}^2)^{-\epsilon}
\end{aligned}
\end{equation}
The $P_{h\perp}$-integrated hadronic tensor is therefore

\begin{equation}
    \begin{aligned}
        \int \dd^{d-2}P_{h\perp} W_{\mu\nu}^{\text{NLO,real}}&=\frac{e_a^2z_h^{1-2\epsilon}}{Q^{2\epsilon}(2\pi)^{3-2\epsilon}}\frac{\pi^{1-\epsilon}}{\Gamma(1-\epsilon)}   \int_{x_B}^{1} \dd x\int_{z_h}^{1}  \frac{\dd z}{z}\\
      &\times \Big(1-\frac{z}{z_h} \Big)^{-\epsilon}\Big(1-\frac{x}{x_B}\Big)^{-\epsilon} w_{\mu\nu}^a\eval_{\chi_T^2=\Big(1-\frac{z}{z_h} \Big)\Big(1-\frac{x}{x_B}\Big)}
    \end{aligned}
\end{equation}
Note that, before performing the transverse momentum integral, we can directly drop the terms linear in $\vec P_{h\perp}$ due to symmetry. For the same reason, we can use the substitution $\vec P_{h\perp}^\mu\vec P_{h\perp}^\nu\to \textcolor{red}{-}\vec P_{h\perp}^2 g_\perp^{\mu\nu}/{(2-2\epsilon)}$. For further manipulation, it is convenient to work with scaled variables defined as
\begin{equation}
    w\equiv\frac{x_B}{x},\qquad v\equiv\frac{z_h}{z} 
\end{equation}
Conveniently, we have that
\begin{equation}
    \int_{x_B}^1\frac{\dd x}{x}=\int_{x_B}^1\frac{\dd w}{w}, \qquad \int_{z_h}^1\frac{\dd z}{z}=\int_{z_h}^1\frac{\dd v}{v}
\end{equation}
therefore 
\begin{equation}
    \begin{aligned}
         \int \dd^{d-2}P_{h\perp} W_{\mu\nu}^{\text{NLO,real}}&=\frac{e_a^2z_h^{1-2\epsilon} x_B}{Q^{2\epsilon}8\pi^2}S_\epsilon    \int_{x_B}^{1} \frac{\dd w}{w}\int_{z_h}^{1}  \frac{\dd v}{v}\frac{1}{w}\\
      &\times \Big(\frac{v-1}{v} \Big)^{-\epsilon}\Big(\frac{w-1}{w}\Big)^{-\epsilon}w_{\mu\nu}^a\eval_{\chi_T^2=(1-\frac{1}{v})(1-\frac{1}{w})}  
    \end{aligned}
\end{equation}
Evidently the integral is not well-behaved for $w,v\to 1$, since it contains non-integrable functions. In order to work with well-behaved quantities and extract the singular behavior of the integral we do the following. After performing the Dirac trace, denoting $f$ and $D$ generic PDFs and FFs respectively, we can write the integrated hadronic tensor in the form
\begin{equation}
    \int \dd^{d-2}P_{h\perp} W_{\mu\nu}^{\text{NLO,real}}= \int_{x_B}^{1} \dd w\int_{z_h}^{1} \dd v \,\,\frac{\hat \sigma_{\mu\nu}(w,v)}{(1-w)^{1+\epsilon}(1-v)^{1+\epsilon}}f(x_B/w)D(z_h/v)
\end{equation}
where $\hat \sigma(w,v)$ is finite in the studied limit. This allows us to extract the $1/\epsilon$ poles and make them manifest in our expressions. We use the useful distribution relation \cite{Schwartz:2014sze}
\begin{equation}
    \frac{1}{(1-w)^{1+\epsilon}}=-\frac{1}{\epsilon}\delta(1-w) + \frac{1}{(1-w)_+}-\epsilon\left(\frac{\ln(1-w)}{1-w}\right)_+ + \mathcal{O}(\epsilon^2)
\end{equation}
where the + prescription is defined by \cite{handbookqcdsterman95}
\begin{equation}
    \int_{x_B}^1 \dd w \,h(w)\left(\frac{g(w)}{1-w}\right)_+=\int _{x_B}^1\dd w \,\frac{h(w)-h(1)}{1-w}g(w) - h(1)\int_0^{x_B}\dd w \frac{g(w)}{1-w}
\end{equation}
Using these relations we derive
\begin{equation}
    \begin{aligned}
        &\int_{x_B}^{1} \dd w\int_{z_h}^{1} \dd v \,\,\frac{\hat \sigma_{\mu\nu}(w,v,\epsilon)}{(1-w)^{1+\epsilon}(1-v)^{1+\epsilon}}f(x_B/w)D(z_h/v)\\&=\int_{x_B}^{1} \dd w\int_{z_h}^{1} \dd v \,\,\Bigg[\frac{1}{\epsilon^2}\delta(1-w)\delta(1-v)-\frac{1}{\epsilon}\frac{\delta(1-w)}{(1-v)_+}-\frac{1}{\epsilon}\frac{\delta(1-v)}{(1-w)_+}\\
        &+\frac{1}{(1-w)_+(1-v)_+}+\left(\frac{\ln(1-w)}{1-w}\right)_+\delta(1-v)+\left(\frac{\ln(1-v)}{1-v}\right)_+\delta(1-w)\Bigg]\\&\times\hat\sigma_{\mu\nu}(w,v)f(x_B/w)D(z_h/v)
    \end{aligned}
\end{equation}
%where the double plus distribution acts in the following way
%\begin{equation}
%\begin{aligned}
%    &\int_{x_B}^{1} \dd w\int_{z_h}^{1} \dd v \,\,\frac{g(w,v)}{((1-w)(1-v))_+}\\
%    &=\int_{x_B}^{1} \dd w\int_{z_h}^{1} \dd v \,\,\frac{g(w,v)-g(1,v)-g(w,1)+g(1,1)}{(1-w)(1-v)}+g(1,1)\ln(1-x_B)\ln(1-z_h)  
%\end{aligned}
%\end{equation}
This procedure makes the $1/\epsilon$ poles manifest and leads to the well-known cancellation of infrared singularities between real and virtual contributions (commonly referred as infrared safety, proved first with the KNL theorem \cite{kinoshita_mass_1962,LeeNauenberg64}). However, the partonic cross typically still exhibits collinear singularities emerging when the parton (coming from the nucleon or fragmenting) becomes collinear with the unobserved parton \cite{hinderer_single-inclusive_2015}. These poles may be absorbed into renormalized parton distribution functions and fragmentation functions \cite{altarelli_large_1979, Collins_2011}. The corresponding poles can be removed in the $\overline{\text{MS}}$ scheme by introducing renormalized functions
\begin{equation}
\begin{aligned}
    f^q_{\text{bare}}(x,\mu)&= f^q_{\text{ren}}(x,\mu)+\frac{\alpha_S}{2\pi}\frac{S_\epsilon}{\epsilon}\left[P_{qq} \otimes f^q_{\text{ren}}+P_{qg} \otimes f^g_{\text{ren}}\right](x,\mu)\\
    D^q_{\text{bare}}(z,\mu)&= D^q_{\text{ren}}(z,\mu)+\frac{\alpha_S}{2\pi}\frac{S_\epsilon}{\epsilon}\left[P_{qq} \otimes D^q_{\text{ren}}+P_{gq} \otimes D^g_{\text{ren}}\right](z,\mu)
\end{aligned}
\end{equation}
with splitting functions
\begin{equation}
    \begin{aligned}
        P_{qq}(y)&=C_F\left[\frac{1+y^2}{(1-y)_+}+\frac{3}{2}\delta(1-y)\right]\\
        P_{qg}(y)&=T_F\left[y^2+(1-y)^2\right]\\
        P_{gq}(y)&=C_F\left[\frac{1+(1-y)^2}{y}\right]
    \end{aligned}
\end{equation}
where the convolution $\otimes$ is a short-hand for
\begin{equation}
    (P\otimes f )(x)\equiv\int_x^1\frac{\dd y}{y} P(y) f\left(\frac{x}{y}\right)
\end{equation}
With such splitting functions, inserting the bare distributions into the LO expression for the cross section leads to additional terms that precisely cancel the collinear poles associated with the observed partons in the NLO partonic cross section. 
\subsubsection{$q\to q$ channel}
After renormalization, the cross section is finite and we may take the $\epsilon\to0$ limit safely. We get the following result (consistent with Appendix C \cite{de_Florian_1998})
\begin{equation}
    \begin{aligned}
        &\frac{\dd \sigma^{q \to q}}{\dd x_B \dd y \dd \phi_l\dd z_h}=\\
        &\frac{2\alpha_{\text{em}}^2}{yQ^2}\left(1-y+\frac{y^2}{2}\right)\int_{x_B}^1\frac{\dd w}{w}\int_{z_h}^1\frac{\dd v}{v}\left[\sum_q e_q^2f_1^q\left(\frac{x_B}{w},\mu\right)D_1^q\left(\frac{z_h}{v},\mu\right)\right]\\
        &\times \Bigg(\left[1+\frac{\alpha_S(\mu)}{2\pi}C_F\left(-8-3\ln\frac{\mu^2}{Q^2}\right)\right]\delta(1-w)\delta(1-v)\\
        &\qquad+\frac{\alpha_S(\mu)}{2\pi}C_F\left[(1+v^2)\left(\frac{\ln(1-v)}{1-v}\right)_+ +1-v+(1+v^2)\frac{\ln v - \ln\frac{\mu^2}{Q^2}}{(1-v)_+}\right]\delta(1-w)\\
        &\qquad+\frac{\alpha_S(\mu)}{2\pi}C_F\left[(1+w^2)\left(\frac{\ln(1-w)}{1-w}\right)_+ +1-w+(1+w^2)\frac{-\ln w - \ln\frac{\mu^2}{Q^2}}{(1-w)_+}\right]\delta(1-v)\\
        &\qquad+\frac{\alpha_S(\mu)}{2\pi}C_F\left[\frac{2v^2w^2-2v^2w-2vw^2+4vw+v^2+w^2-2v-2w+2}{(1-w)_+(1-v)_+}\right]\Bigg)\\
        &+\frac{2\alpha_{\text{em}}^2}{yQ^2}\left(1-y\right)\int_{x_B}^1\frac{\dd w}{w}\int_{z_h}^1\frac{\dd v}{v}\left[\sum_q e_q^2 f_1^q\left(\frac{x_B}{w},\mu\right)D_1^q\left(\frac{z_h}{v},\mu\right)\right]\frac{\alpha_S(\mu)}{2\pi}\left[4C_Fvw\right]
    \end{aligned}
\end{equation}
where both $f_1$ and $D_1$ are the 1-loop renormalized, scale-dependent functions. Interestingly, at NLO also the structure function $(1-y)$ is populated.

\subsubsection{$g \to q$ channel}
Here, we use the convenient mapping between the quark distribution correlator and the gluon distribution correlator \cite{Ji92,Mulders01,Koike_2020}
\begin{equation}
    \Phi_{ij}^a(x)\to\textcolor{red}{\frac{1}{x}}\Phi^{\lambda\eta}_g(x)=\frac{-g_\perp^{\lambda\eta}}{\textcolor{red}{x}(2-2\epsilon)}f_1^g(x)
\end{equation}
We get
\begin{equation}
    \begin{aligned}
        &\frac{\dd \sigma^{g \to q}}{\dd x_B \dd y \dd \phi_l\dd z_h}=\\
        &\frac{2\alpha_{\text{em}}^2}{yQ^2}\left(1-y+\frac{y^2}{2}\right)\int_{x_B}^1\frac{\dd w}{w}\int_{z_h}^1\frac{\dd v}{v} f_1^g\left(\frac{x_B}{w},\mu\right)\left[\sum_q e_q^2D_1^q\left(\frac{z_h}{v},\mu\right)\right]\frac{\alpha_S(\mu)}{2\pi}\\
        &\times T_F\Bigg(\left[(w^2+(1-w)^2)\left(\ln \frac{1-w}{w}-\ln\frac{\mu^2}{Q^2}\right)+2w(1-w)\right]\delta(1-v)\\
        &\qquad+\frac{(w^2+(1-w)^2)(v^2+(1-v)^2)}{v(1-v)_+}\Bigg)\\
        &+\frac{2\alpha_{\text{em}}^2}{yQ^2}\left(1-y\right)\int_{x_B}^1\frac{\dd w}{w}\int_{z_h}^1\frac{\dd v}{v} f_1^g\left(\frac{x_B}{w},\mu\right)\left[\sum_q e_q^2D_1^q\left(\frac{z_h}{v},\mu\right)\right]\frac{\alpha_S(\mu)}{2\pi}\left[8T_Fw(1-w)\right]
    \end{aligned}
\end{equation}

\subsubsection{$q \to g$ channel}
We also use the convenient mapping between the quark fragmentation correlator and the gluon fragmentation correlator
\begin{equation}
    \Delta_{ij}^a(z)\to\textcolor{red}{z}\Delta^{\lambda\eta}_g(z)=\textcolor{red}{z}\frac{-g_T^{\lambda\eta}}{{z^{1-2\epsilon}}}D_1^g(z)
\end{equation}
We get
\begin{equation}
    \begin{aligned}
        &\frac{\dd \sigma^{q \to g}}{\dd x_B \dd y \dd \phi_l\dd z_h}=\\
        &\frac{2\alpha_{\text{em}}^2}{yQ^2}\left(1-y+\frac{y^2}{2}\right)\int_{x_B}^1\frac{\dd w}{w}\int_{z_h}^1\frac{\dd v}{v} \left[\sum_q e_q^2f_1^q\left(\frac{x_B}{w},\mu\right)\right]D_1^g\left(\frac{z_h}{v},\mu\right)\frac{\alpha_S(\mu)}{2\pi}\\
        &\times C_F\Bigg(\left[\frac{1+(1-v)^2}{v}\left(\ln \left(v(1-v)\right)-\ln\frac{\mu^2}{Q^2}\right)+v\right]\delta(1-w)\\
        &\qquad+\frac{1+v^2+w^2-2vw^2-2v^2w+2v^2w^2}{v(1-w)_+}\Bigg)\\
        &+\frac{2\alpha_{\text{em}}^2}{yQ^2}\left(1-y\right)\int_{x_B}^1\frac{\dd w}{w}\int_{z_h}^1\frac{\dd v}{v} \left[\sum_q e_q^2f_1^q\left(\frac{x_B}{w},\mu\right)\right]D_1^g\left(\frac{z_h}{v},\mu\right)\frac{\alpha_S(\mu)}{2\pi}\left[4C_Fw(1-v)\right]
    \end{aligned}
\end{equation}


\begin{equation}
    \begin{aligned}
        \frac{\dd \sigma}{\dd x_B \dd y \dd \phi_l \dd z_h}&=\frac{2 \alpha_{\rm em}^2}{y Q^2} \left(1-y+\frac{y^2}{2}\right)\sum_a e_a^2 \Big(f_1^a \circ \mathcal{C}_U^{a \to a} \circ D_1^a \\
        &\quad +f_1^g \circ \mathcal{C}_{U}^{g \to a} \circ D_1^a+f_1^a \circ \mathcal{C}_U^{a \to g} \circ D_1^g \Big)\\
        &+\frac{2 \alpha_{\rm em}^2}{y Q^2} \left(1-y\right)\sum_a e_a^2 \Big(f_1^a \circ \mathcal{C}_L^{a \to a} \circ D_1^a \\
        &\quad +f_1^g \circ \mathcal{C}_L^{g \to a} \circ D_1^a+f_1^a \circ \mathcal{C}_L^{a \to g} \circ D_1^g \Big)
    \end{aligned}
\end{equation}
\subsection{Numerics}
It is interesting to perform a numerical analysis of the leading-twist NLO cross section presented so far. In order to do so, evidently, one needs to access the relevant PDFs and FFs that show up in the formulae. When it comes to unpolarized PDFs, we use the MSTW 2008 PDF set presented in \cite{Martin_2009}. For unpolarized fragmentation, we employ the DSS FF set \cite{de_Florian_2007}. In Fig.~\ref{fig:f1D1}, plots for $f_1$ and $D_1$ are shown, including different flavours.
\begin{figure}
    \centering
    \subfigure[]{
        \includegraphics[width=0.45\linewidth]{fig/f1.pdf}
    }
        \subfigure[]{
        \includegraphics[width=0.45\linewidth]{fig/D1.pdf}
    }
    \caption{Leading-twist unpolarized $f_1$ PDF and $D_1$ FF as a function of $x_B$ and $z_h$, respectively. Different colors represent different flavors. The scale variation is supposed to give an idea on how much these distributions vary if evaluated at different scales.}
    \label{fig:f1D1}
\end{figure}
It is also interesting to study the leading-twist, unpolarized, triple differential cross section $\dd \sigma/\dd x_B \dd y \dd z_h$ at NLO. A plot of this cross section as a function of the different kinematical variables is shown in Fig.~\ref{fig:plotNLOUUU}.
\begin{figure}
    \centering
    \subfigure[]{
    \includegraphics[width=0.92\textwidth]{fig/sigmaUUU_NLO(x).pdf}
    }
    \subfigure[]{
    \includegraphics[width=0.92\textwidth]{fig/sigmaUUU_NLO(y).pdf}
    }
    \subfigure[]{
    \includegraphics[width=0.92\textwidth]{fig/sigmaUUU_NLO(z).pdf}
    }
    \caption{Unpolarized leading-twist cross section at LO and NLO accuracy as a function of $x_B$ (panel a), $y$ (panel b) and $z_h$ (panel c).}
    \label{fig:plotNLOUUU}
\end{figure} 



\clearpage

\section{Transversely polarized target}
\subsection{Virtual graphs}
\begin{figure}
    \centering
    \includegraphics[width=0.99\linewidth]{fig/VirtNLOTw3.png}
    \caption{Virtual NLO twist-3 test }
    \label{fig:Virt NLO tw3}
    \end{figure}
\subsection{Real graphs}
\begin{figure}
    \centering
    \includegraphics[width=0.99\linewidth]{fig/RealNLOTw3q2qg.png}
    \caption{Real NLO twist-3 $q\to gq $ channel }
    \label{fig:Real NLO tw3}
\end{figure}


