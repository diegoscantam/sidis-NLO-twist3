
\chapter{Hard scattering coefficients}
\section{Unpolarized}
\label{appendix:unpolarized}
The hard scattering coefficients appearing in the unpolarized structure functions are given by
\begin{equation}
    \begin{aligned}
          &\mathcal{C}^{q\to q}_1 (w,v,\mu^2)= \left[C_F\left(-8-3\ln\frac{\mu^2}{Q^2}\right)\right]\delta(1-w)\delta(1-v)\\
        &\qquad+C_F\left[(1+v^2)\left(\frac{\ln(1-v)}{1-v}\right)_+ +1-v+(1+v^2)\frac{\ln v - \ln\frac{\mu^2}{Q^2}}{(1-v)_+}\right]\delta(1-w)\\
        &\qquad+C_F\left[(1+w^2)\left(\frac{\ln(1-w)}{1-w}\right)_+ +1-w+(1+w^2)\frac{-\ln w - \ln\frac{\mu^2}{Q^2}}{(1-w)_+}\right]\delta(1-v)\\
        &\qquad+C_F\left[\frac{2v^2w^2-2v^2w-2vw^2+4vw+v^2+w^2-2v-2w+2}{(1-w)_+(1-v)_+}\right]\\
        &\mathcal{C}^{g\to q}_1 (w,v,\mu^2)=T_F\left[(w^2+(1-w)^2)\left(\ln \frac{1-w}{w}-\ln\frac{\mu^2}{Q^2}\right)+2w(1-w)\right]\delta(1-v)\\
        &\qquad+T_F\left[\frac{(w^2+(1-w)^2)(v^2+(1-v)^2)}{v(1-v)_+}\right]\\
         &\mathcal{C}^{q\to g}_1 (w,v,\mu^2)= C_F\left[\frac{1+(1-v)^2}{v}\left(\ln \left(v(1-v)\right)-\ln\frac{\mu^2}{Q^2}\right)+v\right]\delta(1-w)\\
        &\qquad + C_F\left[\frac{1+v^2+w^2-2vw^2-2v^2w+2v^2w^2}{v(1-w)_+}\right]
    \end{aligned}
\end{equation}
and
\begin{equation}
    \begin{aligned}
          \mathcal{C}^{q\to q}_L (w,v)&= C_F\left[4 vw\right]\\
        \mathcal{C}^{g\to q}_L (w,v)&=T_F\left[8w(1-w)\right]\\
         \mathcal{C}^{q\to g}_L (w,v)&=C_F\left[4w(1-v)\right]
    \end{aligned}
\end{equation}

\section{Transversely polarized nucleon}\label{appendix:polarized}
The hard scattering coefficients appearing in the transverse spin structure functions are given by WRONG!
\begin{equation}
        \begin{aligned}
            \mathcal{C}_{UT}^{q\to qg}&=C_F \Big[\delta(1-w)\delta(1-v)\frac{\left(-15 \zeta +\log ^2(\zeta )-6 \log (\zeta )+2 (2 \zeta +\log (\zeta )-2) \log (Q^2/\mu^2)+15\right)}{2 (\zeta -1)}\\
            &+\delta(1-w)\frac{ \left(\left(\log (Q^2/\mu^2)+\log(1-v)+\log(v)\right) \left(\zeta +2 (\zeta -1) v^2-(\zeta -3) v-1\right)-2 \zeta  v^2+2 v^2-3 v+1\right)}{\zeta  (1-v)_+ }\\
            &+\delta(1-v)\frac{2 w (\log (Q^2/\mu^2)+\log (1-w)-\log (w)-1)}{(1-w)_+}\\
            &-\frac{ w \left((\zeta -1) \left(2 v^2-3 v+1\right) w^2+\zeta  w \left(-\zeta -2 (\zeta -1) v^3-3 v^2+(\zeta +3) v\right)+\zeta ^2 (v-1)^2\right)}{\zeta  (1-v)_+ (1-w)_+ (\zeta  (v-1)+(\zeta -1) w)} \Big]\\
            &+N_C \Big[\delta(1-w)\delta(1-v)\frac{1}{4} \left(-\frac{2 \log (1-\zeta )}{\zeta }-\frac{\log (\zeta ) (\log (\zeta )+2 \log (Q^2/\mu^2)-4)}{\zeta -1}\right)\\
            &+\delta(1-w)\frac{\left(  \left(\log (Q^2/\mu^2)+\log (1-v)+\log(v) \right)\left(\zeta +\left(-\zeta ^2+\zeta -2\right) v+1\right)+2 \zeta ^2 v^2-\zeta ^2 v-\zeta  v \right)}{2 (\zeta -1) \zeta  (\zeta  v-1)}\\
            &+ \frac{ w \left(2 \zeta ^2 (v-1)+(\zeta -1)^2 (2 v-1) w^2-3 \zeta  (\zeta -1) (v-1) w\right)}{2 (1-\zeta) \zeta  (1-w)_+ (\zeta  (v-1)+(\zeta -1) w)}\Big]
        \end{aligned}
    \end{equation}


\chapter{Splitting kernels}
\section{Unpolarized}
\section{Transversely polarized nucleon}


\clearpage

